%\documentclass{../tex/miscdoc}
%\lhead{Project Name}
%\rhead{Literature Review}
%\begin{document}
% --------------------

\textbf{\citet{alvarez2008a}}
\begin{itemize}
	\item \textbf{Cite Key:} alvarez2008a
	\item \textbf{Title​:} The Effect of Voter Identification Laws on Turnout
	\item \textbf{Authors​:} R. Michael Alvarez, Delia Bailey, and Jonathan N. Katz
	\item \textbf{Source Type​:} Working Paper (CalTech Social Science Series)
	\item \textbf{Topic Area​:} Registration/Turnout, Voter ID
	\item \textbf{Main Finding​:} The strictest forms of voter identification requirements have a negative impact on the participation of registered voters relative to the weakest requirement (stating one’s name. The stricter voter identification requirements depress turnout to a greater extent for less educated and lower income populations, for both minorities and non-minorities.
	\item \textbf{Magnitude and Direction​:} see figs 5-9 (individual level analysis)
	\item \textbf{Implications​:} Strict voter ID requirements don’t necessarily negatively impact minority voting to a greater degree, but they do have a negative impact in general and among the less educated and lower income residents. The authors’ finding that strict ID requirements depress turnout has not been replicated, suggesting that one of the bias scenarios related to the multiple stages of voter participation might be at play.
	\item \textbf{Relevant Variables​:}
	\begin{itemize}
		\item IV: voter identification requirements
		\item DV: estimated percentage change in turnout among registered voters at the state-level
		\item Control: education, income, employment, race, age, gender
	\end{itemize}
		\item \textbf{Methodology​:}
	\begin{itemize}
		\item Outcome Measures (calculation/construction): probability of voting
		\item Unit of Analysis: voter (aggregate and individual-level
		\item Model: Bayesian shrinkage estimator, logistic regression
	\end{itemize}
	\item \textbf{References:}
	\item \textbf{Notes:}
\end{itemize}

%\end{document}